\documentclass[]{tufte-handout}

% ams
\usepackage{amssymb,amsmath}

\usepackage{ifxetex,ifluatex}
\usepackage{fixltx2e} % provides \textsubscript
\ifnum 0\ifxetex 1\fi\ifluatex 1\fi=0 % if pdftex
  \usepackage[T1]{fontenc}
  \usepackage[utf8]{inputenc}
\else % if luatex or xelatex
  \makeatletter
  \@ifpackageloaded{fontspec}{}{\usepackage{fontspec}}
  \makeatother
  \defaultfontfeatures{Ligatures=TeX,Scale=MatchLowercase}
  \makeatletter
  \@ifpackageloaded{soul}{
     \renewcommand\allcapsspacing[1]{{\addfontfeature{LetterSpace=15}#1}}
     \renewcommand\smallcapsspacing[1]{{\addfontfeature{LetterSpace=10}#1}}
   }{}
  \makeatother

\fi

% graphix
\usepackage{graphicx}
\setkeys{Gin}{width=\linewidth,totalheight=\textheight,keepaspectratio}

% booktabs
\usepackage{booktabs}

% url
\usepackage{url}

% hyperref
\usepackage{hyperref}

% units.
\usepackage{units}


\setcounter{secnumdepth}{-1}

% citations


% pandoc syntax highlighting
\usepackage{color}
\usepackage{fancyvrb}
\newcommand{\VerbBar}{|}
\newcommand{\VERB}{\Verb[commandchars=\\\{\}]}
\DefineVerbatimEnvironment{Highlighting}{Verbatim}{commandchars=\\\{\}}
% Add ',fontsize=\small' for more characters per line
\newenvironment{Shaded}{}{}
\newcommand{\AlertTok}[1]{\textcolor[rgb]{1.00,0.00,0.00}{\textbf{#1}}}
\newcommand{\AnnotationTok}[1]{\textcolor[rgb]{0.38,0.63,0.69}{\textbf{\textit{#1}}}}
\newcommand{\AttributeTok}[1]{\textcolor[rgb]{0.49,0.56,0.16}{#1}}
\newcommand{\BaseNTok}[1]{\textcolor[rgb]{0.25,0.63,0.44}{#1}}
\newcommand{\BuiltInTok}[1]{#1}
\newcommand{\CharTok}[1]{\textcolor[rgb]{0.25,0.44,0.63}{#1}}
\newcommand{\CommentTok}[1]{\textcolor[rgb]{0.38,0.63,0.69}{\textit{#1}}}
\newcommand{\CommentVarTok}[1]{\textcolor[rgb]{0.38,0.63,0.69}{\textbf{\textit{#1}}}}
\newcommand{\ConstantTok}[1]{\textcolor[rgb]{0.53,0.00,0.00}{#1}}
\newcommand{\ControlFlowTok}[1]{\textcolor[rgb]{0.00,0.44,0.13}{\textbf{#1}}}
\newcommand{\DataTypeTok}[1]{\textcolor[rgb]{0.56,0.13,0.00}{#1}}
\newcommand{\DecValTok}[1]{\textcolor[rgb]{0.25,0.63,0.44}{#1}}
\newcommand{\DocumentationTok}[1]{\textcolor[rgb]{0.73,0.13,0.13}{\textit{#1}}}
\newcommand{\ErrorTok}[1]{\textcolor[rgb]{1.00,0.00,0.00}{\textbf{#1}}}
\newcommand{\ExtensionTok}[1]{#1}
\newcommand{\FloatTok}[1]{\textcolor[rgb]{0.25,0.63,0.44}{#1}}
\newcommand{\FunctionTok}[1]{\textcolor[rgb]{0.02,0.16,0.49}{#1}}
\newcommand{\ImportTok}[1]{#1}
\newcommand{\InformationTok}[1]{\textcolor[rgb]{0.38,0.63,0.69}{\textbf{\textit{#1}}}}
\newcommand{\KeywordTok}[1]{\textcolor[rgb]{0.00,0.44,0.13}{\textbf{#1}}}
\newcommand{\NormalTok}[1]{#1}
\newcommand{\OperatorTok}[1]{\textcolor[rgb]{0.40,0.40,0.40}{#1}}
\newcommand{\OtherTok}[1]{\textcolor[rgb]{0.00,0.44,0.13}{#1}}
\newcommand{\PreprocessorTok}[1]{\textcolor[rgb]{0.74,0.48,0.00}{#1}}
\newcommand{\RegionMarkerTok}[1]{#1}
\newcommand{\SpecialCharTok}[1]{\textcolor[rgb]{0.25,0.44,0.63}{#1}}
\newcommand{\SpecialStringTok}[1]{\textcolor[rgb]{0.73,0.40,0.53}{#1}}
\newcommand{\StringTok}[1]{\textcolor[rgb]{0.25,0.44,0.63}{#1}}
\newcommand{\VariableTok}[1]{\textcolor[rgb]{0.10,0.09,0.49}{#1}}
\newcommand{\VerbatimStringTok}[1]{\textcolor[rgb]{0.25,0.44,0.63}{#1}}
\newcommand{\WarningTok}[1]{\textcolor[rgb]{0.38,0.63,0.69}{\textbf{\textit{#1}}}}

% table with pandoc

% multiplecol
\usepackage{multicol}

% strikeout
\usepackage[normalem]{ulem}

% morefloats
\usepackage{morefloats}


% tightlist macro required by pandoc >= 1.14
\providecommand{\tightlist}{%
  \setlength{\itemsep}{0pt}\setlength{\parskip}{0pt}}

% title / author / date
\title{USING OPEN SOURCE DATA TO VISUALIZE BIOLOGICAL, CHEMICAL, AND
PHYSICAL OCEANOGRAPHIC PROCESSES}
\author{Robert Dellinger}
\date{5/5/2022}


\begin{document}

\maketitle




\begin{Shaded}
\begin{Highlighting}[]
\NormalTok{knitr}\SpecialCharTok{::}\NormalTok{opts\_chunk}\SpecialCharTok{$}\FunctionTok{set}\NormalTok{(}\AttributeTok{warning=}\ConstantTok{FALSE}\NormalTok{, }\AttributeTok{fig.algin=}\StringTok{"center"}\NormalTok{, }\AttributeTok{fig.path =} \StringTok{\textquotesingle{}../output/\textquotesingle{}}\NormalTok{)}

\NormalTok{devtools}\SpecialCharTok{::}\FunctionTok{install\_github}\NormalTok{(}\StringTok{"ropensci/rerddap"}\NormalTok{)}
\end{Highlighting}
\end{Shaded}

\begin{verbatim}
## Skipping install of 'rerddap' from a github remote, the SHA1 (0723cafc) has not changed since last install.
##   Use `force = TRUE` to force installation
\end{verbatim}

\begin{Shaded}
\begin{Highlighting}[]
\FunctionTok{library}\NormalTok{(oce)}
\end{Highlighting}
\end{Shaded}

\begin{verbatim}
## Loading required package: gsw
\end{verbatim}

\begin{Shaded}
\begin{Highlighting}[]
\FunctionTok{library}\NormalTok{(ocedata)}
\FunctionTok{library}\NormalTok{(ncdf4)}
\FunctionTok{library}\NormalTok{(tidyverse)}
\end{Highlighting}
\end{Shaded}

\begin{verbatim}
## -- Attaching packages --------------------------------------- tidyverse 1.3.1 --
\end{verbatim}

\begin{verbatim}
## v ggplot2 3.3.6     v purrr   0.3.4
## v tibble  3.1.7     v dplyr   1.0.9
## v tidyr   1.2.0     v stringr 1.4.0
## v readr   2.1.2     v forcats 0.5.1
\end{verbatim}

\begin{verbatim}
## -- Conflicts ------------------------------------------ tidyverse_conflicts() --
## x dplyr::filter() masks stats::filter()
## x dplyr::lag()    masks stats::lag()
\end{verbatim}

\begin{Shaded}
\begin{Highlighting}[]
\FunctionTok{library}\NormalTok{(lubridate)}
\end{Highlighting}
\end{Shaded}

\begin{verbatim}
## 
## Attaching package: 'lubridate'
\end{verbatim}

\begin{verbatim}
## The following objects are masked from 'package:base':
## 
##     date, intersect, setdiff, union
\end{verbatim}

\begin{Shaded}
\begin{Highlighting}[]
\FunctionTok{library}\NormalTok{(tidync)}
\FunctionTok{library}\NormalTok{(dplyr)}
\FunctionTok{library}\NormalTok{(marmap)}
\end{Highlighting}
\end{Shaded}

\begin{verbatim}
## Registered S3 methods overwritten by 'adehabitatMA':
##   method                       from
##   print.SpatialPixelsDataFrame sp  
##   print.SpatialPixels          sp
\end{verbatim}

\begin{verbatim}
## 
## Attaching package: 'marmap'
\end{verbatim}

\begin{verbatim}
## The following object is masked from 'package:oce':
## 
##     plotProfile
\end{verbatim}

\begin{verbatim}
## The following object is masked from 'package:grDevices':
## 
##     as.raster
\end{verbatim}

\begin{Shaded}
\begin{Highlighting}[]
\FunctionTok{library}\NormalTok{(ggplot2)}
\FunctionTok{library}\NormalTok{(ggmap)}
\end{Highlighting}
\end{Shaded}

\begin{verbatim}
## Google's Terms of Service: https://cloud.google.com/maps-platform/terms/.
\end{verbatim}

\begin{verbatim}
## Please cite ggmap if you use it! See citation("ggmap") for details.
\end{verbatim}

\begin{Shaded}
\begin{Highlighting}[]
\FunctionTok{library}\NormalTok{(gganimate)}
\FunctionTok{library}\NormalTok{(stringr)}
\FunctionTok{library}\NormalTok{(plotly)}
\end{Highlighting}
\end{Shaded}

\begin{verbatim}
## 
## Attaching package: 'plotly'
\end{verbatim}

\begin{verbatim}
## The following object is masked from 'package:ggmap':
## 
##     wind
\end{verbatim}

\begin{verbatim}
## The following object is masked from 'package:ggplot2':
## 
##     last_plot
\end{verbatim}

\begin{verbatim}
## The following object is masked from 'package:stats':
## 
##     filter
\end{verbatim}

\begin{verbatim}
## The following object is masked from 'package:graphics':
## 
##     layout
\end{verbatim}

\begin{Shaded}
\begin{Highlighting}[]
\FunctionTok{library}\NormalTok{(sf)}
\end{Highlighting}
\end{Shaded}

\begin{verbatim}
## Linking to GEOS 3.9.1, GDAL 3.4.0, PROJ 8.1.1; sf_use_s2() is TRUE
\end{verbatim}

\begin{Shaded}
\begin{Highlighting}[]
\FunctionTok{library}\NormalTok{(spData)}
\end{Highlighting}
\end{Shaded}

\begin{verbatim}
## To access larger datasets in this package, install the spDataLarge
## package with: `install.packages('spDataLarge',
## repos='https://nowosad.github.io/drat/', type='source')`
\end{verbatim}

\begin{Shaded}
\begin{Highlighting}[]
\FunctionTok{library}\NormalTok{(rerddap)}
\end{Highlighting}
\end{Shaded}

\begin{verbatim}
## Registered S3 method overwritten by 'hoardr':
##   method           from
##   print.cache_info httr
\end{verbatim}

\begin{Shaded}
\begin{Highlighting}[]
\FunctionTok{library}\NormalTok{(raster)}
\end{Highlighting}
\end{Shaded}

\begin{verbatim}
## Loading required package: sp
\end{verbatim}

\begin{verbatim}
## 
## Attaching package: 'raster'
\end{verbatim}

\begin{verbatim}
## The following object is masked from 'package:plotly':
## 
##     select
\end{verbatim}

\begin{verbatim}
## The following object is masked from 'package:dplyr':
## 
##     select
\end{verbatim}

\begin{Shaded}
\begin{Highlighting}[]
\FunctionTok{library}\NormalTok{(mapdata)}
\end{Highlighting}
\end{Shaded}

\begin{verbatim}
## Loading required package: maps
\end{verbatim}

\begin{verbatim}
## 
## Attaching package: 'maps'
\end{verbatim}

\begin{verbatim}
## The following object is masked from 'package:purrr':
## 
##     map
\end{verbatim}

\hypertarget{mapping-with-the-oce-package}{%
\section{Mapping with the OCE
package}\label{mapping-with-the-oce-package}}

\hypertarget{read-in-datasets}{%
\subsection{Read in Datasets}\label{read-in-datasets}}

The OCE package is a package written by Kelley and Richards (2018) and
is focused on oceanographic mapping and plotting that meets publication
quality standards. Here, we use the mapPlot() function to create maps
using coastline data and visualize oceanographic processes.

\hypertarget{downloading-data}{%
\subsection{Downloading Data}\label{downloading-data}}

First we will read in data from the oce package using the data()
function

\begin{Shaded}
\begin{Highlighting}[]
\NormalTok{Coastline }\OtherTok{\textless{}{-}} \FunctionTok{data}\NormalTok{(}\StringTok{"coastlineWorld"}\NormalTok{) }\CommentTok{\#coastline data}
\NormalTok{Coastline\_Medium }\OtherTok{\textless{}{-}} \FunctionTok{data}\NormalTok{(}\StringTok{"coastlineWorldMedium"}\NormalTok{) }\CommentTok{\#medium{-}scale coastline data}
\NormalTok{Coastline\_Fine }\OtherTok{\textless{}{-}} \FunctionTok{data}\NormalTok{(}\StringTok{"coastlineWorldFine"}\NormalTok{) }\CommentTok{\#fine{-}scale{-}scale coastline data}
\NormalTok{Bathymetry }\OtherTok{\textless{}{-}} \FunctionTok{as.topo}\NormalTok{(}\FunctionTok{getNOAA.bathy}\NormalTok{(}\SpecialCharTok{{-}}\DecValTok{180}\NormalTok{, }\DecValTok{0}\NormalTok{, }\DecValTok{0}\NormalTok{, }\DecValTok{90}\NormalTok{)) }\CommentTok{\#coordinates for bathymetry data from NOAA (Northern Hemisphere)}
\end{Highlighting}
\end{Shaded}

\begin{verbatim}
## Querying NOAA database ...
\end{verbatim}

\begin{verbatim}
## This may take seconds to minutes, depending on grid size
\end{verbatim}

\begin{verbatim}
## Building bathy matrix ...
\end{verbatim}

\hypertarget{creating-a-map}{%
\section{Creating a Map}\label{creating-a-map}}

Using the map from the oce package we will create a simple map function
that can be rerun

\begin{Shaded}
\begin{Highlighting}[]
\DocumentationTok{\#\# setting the map limits}
\FunctionTok{par}\NormalTok{(}\AttributeTok{mar=}\FunctionTok{c}\NormalTok{(}\DecValTok{2}\NormalTok{, }\DecValTok{2}\NormalTok{, }\DecValTok{1}\NormalTok{, }\DecValTok{1}\NormalTok{)) }\CommentTok{\# margins}
\NormalTok{lonlim }\OtherTok{=} \FunctionTok{c}\NormalTok{(}\SpecialCharTok{{-}}\DecValTok{126}\NormalTok{,}\SpecialCharTok{{-}}\DecValTok{116}\NormalTok{) }\CommentTok{\# longitude}
\NormalTok{latlim }\OtherTok{=} \FunctionTok{c}\NormalTok{(}\DecValTok{32}\NormalTok{,}\DecValTok{40}\NormalTok{) }\CommentTok{\# latitude}

\NormalTok{coastline.map }\OtherTok{\textless{}{-}} \ControlFlowTok{function}\NormalTok{() \{ }
  \FunctionTok{mapPlot}\NormalTok{(coastlineWorldFine, }\CommentTok{\#using fine{-}scale data}
        \AttributeTok{projection=}\StringTok{"+proj=mill"}\NormalTok{, }\CommentTok{\#setting the projection type}
        \AttributeTok{col=}\StringTok{"grey"}\NormalTok{, }\CommentTok{\#setting colors,}
        \AttributeTok{longitudelim=}\NormalTok{lonlim, }
        \AttributeTok{latitudelim=}\NormalTok{latlim,}
        \AttributeTok{lonlabels=}\ConstantTok{TRUE}\NormalTok{,}
        \AttributeTok{latlabels=}\ConstantTok{TRUE}\NormalTok{,}
        \AttributeTok{geographical=}\DecValTok{4}\NormalTok{,}
        \AttributeTok{clip=}\ConstantTok{TRUE}\NormalTok{)}
\NormalTok{\}}

\FunctionTok{coastline.map}\NormalTok{()}
\end{Highlighting}
\end{Shaded}

\begin{figure}
\includegraphics{../output/Creating a Map Function Using OCE-1} \caption[Map of California]{Map of California}\label{fig:Creating a Map Function Using OCE}
\end{figure}

\hypertarget{creating-a-map-with-bathymetry}{%
\section{Creating a Map with
Bathymetry}\label{creating-a-map-with-bathymetry}}

We will now create a map that mimics the depth of the ocean floor using
our previous map function. The provided color palettes in the package
can be used to visualize oceanographic phenomena such as bathymetry.

\begin{Shaded}
\begin{Highlighting}[]
\DocumentationTok{\#\# setting the map limits}
\FunctionTok{par}\NormalTok{(}\AttributeTok{mar=}\FunctionTok{c}\NormalTok{(}\DecValTok{2}\NormalTok{, }\DecValTok{2}\NormalTok{, }\DecValTok{1}\NormalTok{, }\DecValTok{1}\NormalTok{)) }\CommentTok{\# margins}
\NormalTok{lonlim }\OtherTok{=} \FunctionTok{c}\NormalTok{(}\SpecialCharTok{{-}}\DecValTok{125}\NormalTok{,}\SpecialCharTok{{-}}\DecValTok{116}\NormalTok{) }\CommentTok{\# longitude}
\NormalTok{latlim }\OtherTok{=} \FunctionTok{c}\NormalTok{(}\FloatTok{33.5}\NormalTok{,}\FloatTok{39.5}\NormalTok{) }\CommentTok{\# latitude}

\CommentTok{\#Creating a Map with Bathymetry}
\CommentTok{\#Place Map}
\FunctionTok{coastline.map}\NormalTok{()}
\CommentTok{\#Overlaying Bathymetry Data}
\FunctionTok{mapImage}\NormalTok{(Bathymetry, }\AttributeTok{col=}\NormalTok{oceColorsGebco, }\AttributeTok{breaks=}\FunctionTok{seq}\NormalTok{(}\SpecialCharTok{{-}}\DecValTok{6000}\NormalTok{, }\DecValTok{0}\NormalTok{, }\DecValTok{100}\NormalTok{))}
\FunctionTok{mapImage}\NormalTok{(Bathymetry, }\AttributeTok{col=}\NormalTok{oceColorsGebco, }\AttributeTok{breaks=}\FunctionTok{seq}\NormalTok{(}\SpecialCharTok{{-}}\DecValTok{6000}\NormalTok{, }\DecValTok{0}\NormalTok{, }\DecValTok{500}\NormalTok{), }\AttributeTok{filledContour =} \ConstantTok{TRUE}\NormalTok{)}
\CommentTok{\#Adding in Coastline}
\FunctionTok{mapPolygon}\NormalTok{(coastlineWorldFine, }\AttributeTok{col =} \StringTok{"darkolivegreen3"}\NormalTok{, }\AttributeTok{border =} \StringTok{"black"}\NormalTok{) }
\CommentTok{\#Adding in Grid}
\FunctionTok{mapGrid}\NormalTok{(}\AttributeTok{dlongitude=}\DecValTok{1}\NormalTok{, }\AttributeTok{dlatitude=}\DecValTok{1}\NormalTok{, }\AttributeTok{col=}\StringTok{"black"}\NormalTok{, }\AttributeTok{lty=}\DecValTok{3}\NormalTok{) }

\CommentTok{\#Setting points on a map with names and the respective locations}
\NormalTok{sites }\OtherTok{=} \FunctionTok{data.frame}\NormalTok{(}\AttributeTok{name =} \FunctionTok{c}\NormalTok{(}\StringTok{"Leo Carrillo State Beach }\SpecialCharTok{\textbackslash{}n}\StringTok{ Malibu, CA"}\NormalTok{, }\StringTok{"Bodega Marine Lab }\SpecialCharTok{\textbackslash{}n}\StringTok{ Bodega Bay, CA"}\NormalTok{), }
                      \AttributeTok{lon =} \FunctionTok{c}\NormalTok{(}\SpecialCharTok{{-}}\FloatTok{118.9279}\NormalTok{, }\SpecialCharTok{{-}}\FloatTok{123.0481}\NormalTok{), }\CommentTok{\#locations and their respective longitude and latitudes}
                      \AttributeTok{lat =} \FunctionTok{c}\NormalTok{(}\FloatTok{34.0622}\NormalTok{, }\FloatTok{38.3332}\NormalTok{)) }
\CommentTok{\#Placing points on the map}
\FunctionTok{mapPoints}\NormalTok{(}\AttributeTok{longitude =}\NormalTok{ sites}\SpecialCharTok{$}\NormalTok{lon, }\CommentTok{\#placing points on map using longitude}
          \AttributeTok{latitude =}\NormalTok{ sites}\SpecialCharTok{$}\NormalTok{lat, }\CommentTok{\#placing points on map using latitude}
          \AttributeTok{pch =} \DecValTok{20}\NormalTok{, }
          \AttributeTok{cex =} \FloatTok{1.75}\NormalTok{)  }\CommentTok{\#point size}
\CommentTok{\#Overlaying text}
\FunctionTok{mapText}\NormalTok{(}\AttributeTok{longitude =}\NormalTok{ sites}\SpecialCharTok{$}\NormalTok{lon}\FloatTok{+1.5}\NormalTok{, }\CommentTok{\#map text location long and lat}
        \AttributeTok{latitude =}\NormalTok{ sites}\SpecialCharTok{$}\NormalTok{lat}\FloatTok{+.75}\NormalTok{, }
        \AttributeTok{labels =}\NormalTok{ sites}\SpecialCharTok{$}\NormalTok{name,}
        \AttributeTok{cex =} \FloatTok{1.2}\NormalTok{, }\CommentTok{\#font size}
        \AttributeTok{col=}\StringTok{"black"}\NormalTok{) }\CommentTok{\# changing font family}
\end{Highlighting}
\end{Shaded}

\begin{figure}
\includegraphics{../output/unnamed-chunk-1-1} \caption[Map of California with Bathymetry]{Map of California with Bathymetry}\label{fig:unnamed-chunk-1}
\end{figure}

\hypertarget{creating-a-map-with-overlayed-sst-data}{%
\section{Creating a Map with Overlayed SST
Data}\label{creating-a-map-with-overlayed-sst-data}}

We will now create a map that mimics that shows us the sea surface
temperature of the ocean and use the provided temeprature color palette
to visualize differences in ocean temperature. NOAA and NASA provide
open source data for marine biology and oceanography data that can be
used for creating visuals. For this portion we will use data downloaded
from the Group for High Resolution Sea Surface Temperature (GHRSST)
which has entire data sets of regional and global data linked here
{[}\url{https://www.ghrsst.org/}{]}. The dataset is provided in a netcdf
format so we will have to first read in the data using the nc\_open
function from the netcdf package (Pierce 2017) prior to graphing.

We will download sea surface temperature (SST) analysis data provided by
NOAA which provides a combination of daily satellite and in situ
temperature temperature measurements on a global level. We will use data
from August 1, 2018, the day of the hottest observed SST by Scripps
Institution of Oceanography since the previous record set in 1931.

\hypertarget{reading-in-open-source-sst-data-from-noaa-and-nasa}{%
\subsection{Reading in Open Source SST Data from NOAA and
NASA}\label{reading-in-open-source-sst-data-from-noaa-and-nasa}}

\begin{Shaded}
\begin{Highlighting}[]
\DocumentationTok{\#\# Opening downloaded data from NOAA }
\NormalTok{sst.data }\OtherTok{\textless{}{-}} \FunctionTok{nc\_open}\NormalTok{(}\StringTok{"/Users/robertdellinger/Documents/Independent\_Project\_Dellinger/Data/SST\_08\_01\_2018.nc"}\NormalTok{)}

\CommentTok{\# extract the longitude variable}
\NormalTok{sst.lon }\OtherTok{=} \FunctionTok{ncvar\_get}\NormalTok{(sst.data, }\StringTok{"lon"}\NormalTok{)}
\CommentTok{\# extract the latitude variable}
\NormalTok{sst.lat }\OtherTok{=} \FunctionTok{ncvar\_get}\NormalTok{(sst.data, }\StringTok{"lat"}\NormalTok{)}
\CommentTok{\# extract sst variable}
\NormalTok{sst }\OtherTok{=} \FunctionTok{ncvar\_get}\NormalTok{(sst.data, }\StringTok{"analysed\_sst"}\NormalTok{)}
\CommentTok{\# extract time }

\DocumentationTok{\#\# Kelvin to Degree celsius and calibrate}
\NormalTok{sst }\OtherTok{=}\NormalTok{ sst}\FloatTok{{-}273.149993896484}
\end{Highlighting}
\end{Shaded}

\hypertarget{hottest-observed-sst-for-the-southern-california-region}{%
\section{Hottest Observed SST for the Southern California
Region}\label{hottest-observed-sst-for-the-southern-california-region}}

\begin{Shaded}
\begin{Highlighting}[]
\DocumentationTok{\#\# setting the map limits}
\FunctionTok{par}\NormalTok{(}\AttributeTok{mar=}\FunctionTok{c}\NormalTok{(}\DecValTok{2}\NormalTok{, }\DecValTok{2}\NormalTok{, }\DecValTok{1}\NormalTok{, }\DecValTok{1}\NormalTok{)) }\CommentTok{\# margins}
\NormalTok{lonlim }\OtherTok{=} \FunctionTok{c}\NormalTok{(}\SpecialCharTok{{-}}\FloatTok{122.25}\NormalTok{,}\SpecialCharTok{{-}}\FloatTok{116.25}\NormalTok{) }\CommentTok{\# longitude}
\NormalTok{latlim }\OtherTok{=} \FunctionTok{c}\NormalTok{(}\FloatTok{32.25}\NormalTok{,}\FloatTok{35.25}\NormalTok{) }\CommentTok{\# latitude}
\NormalTok{templim }\OtherTok{=} \FunctionTok{c}\NormalTok{(}\DecValTok{10}\NormalTok{,}\DecValTok{26}\NormalTok{) }\CommentTok{\#temperature limits in celsisus}

\DocumentationTok{\#\# draw palette (must be done before plotting)}
\FunctionTok{drawPalette}\NormalTok{(}\AttributeTok{zlim =}\NormalTok{ templim, }
            \AttributeTok{zlab =} \StringTok{"Temperature (°C)"}\NormalTok{,}
            \AttributeTok{col=}\NormalTok{oce.colorsJet, }\CommentTok{\#using the oce package temperature colors}
            \AttributeTok{at =} \FunctionTok{seq}\NormalTok{(}\DecValTok{10}\NormalTok{,}\DecValTok{26}\NormalTok{,}\DecValTok{1}\NormalTok{), }
            \AttributeTok{pos =} \DecValTok{1}\NormalTok{)}

\CommentTok{\#adding in sst to overlay the map function}
\FunctionTok{coastline.map}\NormalTok{()}
\FunctionTok{mapImage}\NormalTok{(}\AttributeTok{longitude =}\NormalTok{ sst.lon, }\AttributeTok{latitude =}\NormalTok{ sst.lat, }\AttributeTok{z =}\NormalTok{ sst, }\AttributeTok{zlim =}\NormalTok{ templim, }\AttributeTok{col =} \FunctionTok{oceColorsJet}\NormalTok{(}\DecValTok{120}\NormalTok{))}
\FunctionTok{mapImage}\NormalTok{(}\AttributeTok{longitude =}\NormalTok{ sst.lon, }\AttributeTok{latitude =}\NormalTok{ sst.lat, }\AttributeTok{z =}\NormalTok{ sst, }\AttributeTok{zlim =}\NormalTok{ templim, }\AttributeTok{col =} \FunctionTok{oceColorsJet}\NormalTok{(}\DecValTok{120}\NormalTok{), }\AttributeTok{filledContour =} \ConstantTok{TRUE}\NormalTok{)  }\CommentTok{\#overlay sst \& contour}
\FunctionTok{mapContour}\NormalTok{(}\AttributeTok{longitude =}\NormalTok{ sst.lon, }\AttributeTok{latitude =}\NormalTok{ sst.lat, }\AttributeTok{z =}\NormalTok{ sst, }\AttributeTok{levels =} \FunctionTok{seq}\NormalTok{(}\DecValTok{10}\NormalTok{,}\DecValTok{26}\NormalTok{,}\DecValTok{1}\NormalTok{),}
            \AttributeTok{col =} \StringTok{"black"}\NormalTok{, }\AttributeTok{lty =} \DecValTok{1}\NormalTok{, }\AttributeTok{lwd =} \DecValTok{1}\NormalTok{)}
\FunctionTok{mapPolygon}\NormalTok{(coastlineWorldFine, }\AttributeTok{col=}\StringTok{"darkolivegreen3"}\NormalTok{) }\CommentTok{\# overlay green topography}
\FunctionTok{mapGrid}\NormalTok{(}\AttributeTok{dlongitude=}\DecValTok{1}\NormalTok{, }\AttributeTok{dlatitude=}\DecValTok{1}\NormalTok{, }\AttributeTok{col=}\StringTok{"black"}\NormalTok{, }\AttributeTok{lty=}\DecValTok{3}\NormalTok{) }\DocumentationTok{\#\# adding a map grid}

\CommentTok{\#Setting points on a map with names and the respective locations}
\NormalTok{site }\OtherTok{=} \FunctionTok{data.frame}\NormalTok{(}\AttributeTok{name =} \FunctionTok{c}\NormalTok{(}\StringTok{"Leo Carrillo State Beach }\SpecialCharTok{\textbackslash{}n}\StringTok{ Malibu, CA"}\NormalTok{), }
                      \AttributeTok{lon =} \SpecialCharTok{{-}}\FloatTok{118.9279}\NormalTok{, }\CommentTok{\#locations and their respective longitude and latitudes}
                      \AttributeTok{lat =} \FloatTok{34.0622}\NormalTok{)}
\CommentTok{\#Placing points on the map}
\FunctionTok{mapPoints}\NormalTok{(}\AttributeTok{longitude =}\NormalTok{ site}\SpecialCharTok{$}\NormalTok{lon, }\CommentTok{\#placing points on map using longitude}
          \AttributeTok{latitude =}\NormalTok{ site}\SpecialCharTok{$}\NormalTok{lat, }\CommentTok{\#placing points on map using latitude}
          \AttributeTok{pch =} \DecValTok{20}\NormalTok{, }
          \AttributeTok{cex =} \FloatTok{1.75}\NormalTok{)  }\CommentTok{\#point size}
\CommentTok{\#Overlaying text}
\FunctionTok{mapText}\NormalTok{(}\AttributeTok{longitude =}\NormalTok{ site}\SpecialCharTok{$}\NormalTok{lon}\FloatTok{+1.5}\NormalTok{, }\CommentTok{\#map text location long and lat}
        \AttributeTok{latitude =}\NormalTok{ site}\SpecialCharTok{$}\NormalTok{lat}\FloatTok{+.75}\NormalTok{, }
        \AttributeTok{labels =}\NormalTok{ site}\SpecialCharTok{$}\NormalTok{name,}
        \AttributeTok{cex =} \FloatTok{1.2}\NormalTok{, }\CommentTok{\#font size}
        \AttributeTok{col=}\StringTok{"black"}\NormalTok{) }\CommentTok{\# changing font family}
\end{Highlighting}
\end{Shaded}

\begin{figure}
\includegraphics{../output/unnamed-chunk-2-1} \caption[SST Map for Souththern California]{SST Map for Souththern California}\label{fig:unnamed-chunk-2}
\end{figure}

\hypertarget{hottest-observed-sst-for-the-northern-california-region}{%
\section{Hottest Observed SST for the Northern California
Region}\label{hottest-observed-sst-for-the-northern-california-region}}

\begin{Shaded}
\begin{Highlighting}[]
\DocumentationTok{\#\# setting the map limits}
\FunctionTok{par}\NormalTok{(}\AttributeTok{mar=}\FunctionTok{c}\NormalTok{(}\DecValTok{2}\NormalTok{, }\DecValTok{2}\NormalTok{, }\DecValTok{1}\NormalTok{, }\DecValTok{1}\NormalTok{)) }\CommentTok{\# margins}
\NormalTok{lonlim }\OtherTok{=} \FunctionTok{c}\NormalTok{(}\SpecialCharTok{{-}}\FloatTok{126.25}\NormalTok{,}\SpecialCharTok{{-}}\FloatTok{120.25}\NormalTok{) }\CommentTok{\# longitude}
\NormalTok{latlim }\OtherTok{=} \FunctionTok{c}\NormalTok{(}\FloatTok{37.25}\NormalTok{,}\FloatTok{40.25}\NormalTok{) }\CommentTok{\# latitude }
\NormalTok{templim }\OtherTok{=} \FunctionTok{c}\NormalTok{(}\DecValTok{10}\NormalTok{,}\DecValTok{26}\NormalTok{) }\CommentTok{\#temperature limits in celsisus}

\DocumentationTok{\#\# draw palette (must be done before plotting)}
\FunctionTok{drawPalette}\NormalTok{(}\AttributeTok{zlim =}\NormalTok{ templim, }
            \AttributeTok{zlab =} \StringTok{"Temperature (°C)"}\NormalTok{,}
            \AttributeTok{col=}\NormalTok{oce.colorsJet, }\CommentTok{\#using the oce package temperature colors}
            \AttributeTok{at =} \FunctionTok{seq}\NormalTok{(}\DecValTok{10}\NormalTok{,}\DecValTok{26}\NormalTok{,}\DecValTok{1}\NormalTok{), }
            \AttributeTok{pos =} \DecValTok{1}\NormalTok{)}

\CommentTok{\#adding in sst to overlay the map function}
\FunctionTok{coastline.map}\NormalTok{()}
\FunctionTok{mapImage}\NormalTok{(}\AttributeTok{longitude =}\NormalTok{ sst.lon, }\AttributeTok{latitude =}\NormalTok{ sst.lat, }\AttributeTok{z =}\NormalTok{ sst, }\AttributeTok{zlim =}\NormalTok{ templim, }\AttributeTok{col =} \FunctionTok{oceColorsJet}\NormalTok{(}\DecValTok{120}\NormalTok{))}
\FunctionTok{mapImage}\NormalTok{(}\AttributeTok{longitude =}\NormalTok{ sst.lon, }\AttributeTok{latitude =}\NormalTok{ sst.lat, }\AttributeTok{z =}\NormalTok{ sst, }\AttributeTok{zlim =}\NormalTok{ templim, }\AttributeTok{col =} \FunctionTok{oceColorsJet}\NormalTok{(}\DecValTok{120}\NormalTok{), }\AttributeTok{filledContour =} \ConstantTok{TRUE}\NormalTok{)  }\CommentTok{\#overlay sst \& contour}
\FunctionTok{mapContour}\NormalTok{(}\AttributeTok{longitude =}\NormalTok{ sst.lon, }\AttributeTok{latitude =}\NormalTok{ sst.lat, }\AttributeTok{z =}\NormalTok{ sst, }\AttributeTok{levels =} \FunctionTok{seq}\NormalTok{(}\DecValTok{10}\NormalTok{,}\DecValTok{26}\NormalTok{,}\DecValTok{1}\NormalTok{),}
            \AttributeTok{col =} \StringTok{"black"}\NormalTok{, }\AttributeTok{lty =} \DecValTok{1}\NormalTok{, }\AttributeTok{lwd =} \DecValTok{1}\NormalTok{)}
\FunctionTok{mapPolygon}\NormalTok{(coastlineWorldFine, }\AttributeTok{col=}\StringTok{"darkolivegreen3"}\NormalTok{) }\CommentTok{\# overlay green topography}
\FunctionTok{mapGrid}\NormalTok{(}\AttributeTok{dlongitude=}\DecValTok{1}\NormalTok{, }\AttributeTok{dlatitude=}\DecValTok{1}\NormalTok{, }\AttributeTok{col=}\StringTok{"black"}\NormalTok{, }\AttributeTok{lty=}\DecValTok{3}\NormalTok{) }\DocumentationTok{\#\# adding a map grid}


\CommentTok{\#Setting points on a map with names and the respective locations}
\NormalTok{site }\OtherTok{=} \FunctionTok{data.frame}\NormalTok{(}\AttributeTok{name =} \FunctionTok{c}\NormalTok{(}\StringTok{"Bodega Marine Lab }\SpecialCharTok{\textbackslash{}n}\StringTok{ Bodega Bay, CA"}\NormalTok{), }
                      \AttributeTok{lon =} \SpecialCharTok{{-}}\FloatTok{123.0481}\NormalTok{, }\CommentTok{\#locations and their respective longitude and latitudes}
                      \AttributeTok{lat =} \FloatTok{38.3332}\NormalTok{) }
\CommentTok{\#Placing points on the map}
\FunctionTok{mapPoints}\NormalTok{(}\AttributeTok{longitude =}\NormalTok{ site}\SpecialCharTok{$}\NormalTok{lon, }\CommentTok{\#placing points on map using longitude}
          \AttributeTok{latitude =}\NormalTok{ site}\SpecialCharTok{$}\NormalTok{lat, }\CommentTok{\#placing points on map using latitude}
          \AttributeTok{pch =} \DecValTok{20}\NormalTok{, }
          \AttributeTok{cex =} \FloatTok{1.75}\NormalTok{)  }\CommentTok{\#point size}
\CommentTok{\#Overlaying text}
\FunctionTok{mapText}\NormalTok{(}\AttributeTok{longitude =}\NormalTok{ site}\SpecialCharTok{$}\NormalTok{lon}\FloatTok{+1.5}\NormalTok{, }\CommentTok{\#map text location long and lat}
        \AttributeTok{latitude =}\NormalTok{ site}\SpecialCharTok{$}\NormalTok{lat}\FloatTok{+.75}\NormalTok{, }
        \AttributeTok{labels =}\NormalTok{ site}\SpecialCharTok{$}\NormalTok{name,}
        \AttributeTok{cex =} \FloatTok{1.2}\NormalTok{, }\CommentTok{\#font size}
        \AttributeTok{col=}\StringTok{"black"}\NormalTok{) }\CommentTok{\# changing font family}
\end{Highlighting}
\end{Shaded}

\begin{figure}
\includegraphics{../output/unnamed-chunk-3-1} \caption[SST Map for Northern California]{SST Map for Northern California}\label{fig:unnamed-chunk-3}
\end{figure}

\hypertarget{visualizing-time-series-data}{%
\section{Visualizing Time Series
Data}\label{visualizing-time-series-data}}

Data was downloaded from NOAA Environmental Research Division Data
Access Program (EDRAPP) at a Southern California Coastal Ocean
Observation Station (SCCOOS) linked here
{[}\url{https://sccoos.org/}{]}. Data was downloaded from the station at
the Santa Monica Pier. This data provides us with the SST measurements
taken at frequent intervals for the nearshore coastal ocean of the Los
Angeles region.

Below we read in the packages using the nc\_open() functionand extract
time and SST variables from the data set

\hypertarget{load-in-data}{%
\subsection{Load in Data}\label{load-in-data}}

\begin{Shaded}
\begin{Highlighting}[]
\NormalTok{sccoss.data }\OtherTok{\textless{}{-}} \FunctionTok{read\_csv}\NormalTok{(}\StringTok{"/Users/robertdellinger/Documents/Independent\_Project\_Dellinger/Data/SCCOSS\_SST\_LA\_01012012\_01012022.csv"}\NormalTok{)}
\end{Highlighting}
\end{Shaded}

\begin{verbatim}
## Rows: 796962 Columns: 2
## -- Column specification --------------------------------------------------------
## Delimiter: ","
## chr (2): time, temperature
## 
## i Use `spec()` to retrieve the full column specification for this data.
## i Specify the column types or set `show_col_types = FALSE` to quiet this message.
\end{verbatim}

\begin{Shaded}
\begin{Highlighting}[]
\FunctionTok{names}\NormalTok{(sccoss.data) }\OtherTok{\textless{}{-}} \FunctionTok{c}\NormalTok{(}\StringTok{"datetime"}\NormalTok{,}\StringTok{"SST"}\NormalTok{)}
\end{Highlighting}
\end{Shaded}

\hypertarget{data-wrangling}{%
\section{Data Wrangling}\label{data-wrangling}}

\begin{Shaded}
\begin{Highlighting}[]
\CommentTok{\#clean the data}
\NormalTok{sccoss.data.clean  }\OtherTok{\textless{}{-}}\NormalTok{ sccoss.data[}\SpecialCharTok{{-}}\DecValTok{1}\NormalTok{,] }\SpecialCharTok{\%\textgreater{}\%} \CommentTok{\#remove first column}
  \FunctionTok{mutate}\NormalTok{(}\AttributeTok{datetime =} \FunctionTok{str\_remove}\NormalTok{(datetime, }\StringTok{"Z"}\NormalTok{), }\CommentTok{\#remove z from string}
         \AttributeTok{date =} \FunctionTok{as.Date}\NormalTok{(}\FunctionTok{ymd}\NormalTok{(}\FunctionTok{str\_split\_fixed}\NormalTok{(datetime, }\StringTok{"T"}\NormalTok{, }\AttributeTok{n=}\ConstantTok{Inf}\NormalTok{)[,}\DecValTok{1}\NormalTok{])), }\CommentTok{\#split string into date}
         \AttributeTok{time=} \FunctionTok{str\_split\_fixed}\NormalTok{(datetime, }\StringTok{"T"}\NormalTok{, }\AttributeTok{n=}\ConstantTok{Inf}\NormalTok{)[,}\DecValTok{2}\NormalTok{], }\CommentTok{\#split string into time}
         \AttributeTok{SST =} \FunctionTok{as.numeric}\NormalTok{(SST)) }\SpecialCharTok{\%\textgreater{}\%} \CommentTok{\#change sst to a numeric value}
  \FunctionTok{filter}\NormalTok{(SST}\SpecialCharTok{\textgreater{}}\DecValTok{10}\NormalTok{,}
\NormalTok{         SST}\SpecialCharTok{\textless{}}\DecValTok{30}\NormalTok{) }\SpecialCharTok{\%\textgreater{}\%}
\NormalTok{  dplyr}\SpecialCharTok{::}\FunctionTok{select}\NormalTok{(}\StringTok{"date"}\NormalTok{, }\StringTok{"SST"}\NormalTok{) }

\NormalTok{sccoss.summary }\OtherTok{\textless{}{-}}\NormalTok{ sccoss.data.clean }\SpecialCharTok{\%\textgreater{}\%} 
  \FunctionTok{group\_by}\NormalTok{(date)}\SpecialCharTok{\%\textgreater{}\%} \CommentTok{\# group date}
  \FunctionTok{summarise}\NormalTok{(}\AttributeTok{SST.mean =} \FunctionTok{mean}\NormalTok{(SST, }\AttributeTok{na.rm=}\ConstantTok{TRUE}\NormalTok{), }\CommentTok{\#calculate mean}
            \AttributeTok{SST.sd =} \FunctionTok{sd}\NormalTok{(SST, }\AttributeTok{na.rm=}\ConstantTok{TRUE}\NormalTok{),}
            \AttributeTok{SST.min =} \FunctionTok{min}\NormalTok{(SST, }\AttributeTok{na.rm=}\ConstantTok{TRUE}\NormalTok{),}
            \AttributeTok{SST.max =} \FunctionTok{max}\NormalTok{(SST, }\AttributeTok{na.rm=}\ConstantTok{TRUE}\NormalTok{))}

\NormalTok{sccoss.dataset }\OtherTok{\textless{}{-}} \FunctionTok{full\_join}\NormalTok{(sccoss.data.clean, sccoss.summary, }\AttributeTok{by=}\StringTok{"date"}\NormalTok{) }\CommentTok{\#join data sets}

\NormalTok{sccoss.timeseries }\OtherTok{\textless{}{-}}\NormalTok{ sccoss.dataset }\SpecialCharTok{\%\textgreater{}\%} 
  \FunctionTok{filter}\NormalTok{(date }\SpecialCharTok{\textless{}} \FunctionTok{ymd}\NormalTok{(}\StringTok{"2022{-}01{-}01"}\NormalTok{), }\CommentTok{\#filter years}
\NormalTok{         date }\SpecialCharTok{\textgreater{}} \FunctionTok{ymd}\NormalTok{(}\StringTok{"2019{-}01{-}01"}\NormalTok{))}
\end{Highlighting}
\end{Shaded}

\#Plotting Time Series Temperature Data

\begin{Shaded}
\begin{Highlighting}[]
\CommentTok{\# make the plot and add labels}
\NormalTok{sst.plot }\OtherTok{\textless{}{-}} \FunctionTok{ggplot}\NormalTok{(sccoss.timeseries, }\FunctionTok{aes}\NormalTok{(}\AttributeTok{x=}\NormalTok{date, }\AttributeTok{y=}\NormalTok{SST.mean)) }\SpecialCharTok{+} \CommentTok{\#date by mean sst}
  \FunctionTok{geom\_line}\NormalTok{(}\FunctionTok{aes}\NormalTok{(}\AttributeTok{x=}\NormalTok{date, }\AttributeTok{y=}\NormalTok{SST.mean), }\AttributeTok{color=}\StringTok{"darkcyan"}\NormalTok{, }\AttributeTok{alpha=}\FloatTok{0.5}\NormalTok{) }\SpecialCharTok{+}
  \FunctionTok{geom\_smooth}\NormalTok{(}\AttributeTok{span=}\FloatTok{0.25}\NormalTok{, }\AttributeTok{se=}\ConstantTok{FALSE}\NormalTok{, }\AttributeTok{color=}\StringTok{"darkcyan"}\NormalTok{) }\SpecialCharTok{+}
  \FunctionTok{geom\_ribbon}\NormalTok{(}\FunctionTok{aes}\NormalTok{(}\AttributeTok{ymin=}\NormalTok{SST.min, }\AttributeTok{ymax=}\NormalTok{SST.max), }\AttributeTok{fill=}\StringTok{"darkcyan"}\NormalTok{, }\AttributeTok{alpha=}\FloatTok{0.1}\NormalTok{)}\SpecialCharTok{+}
  \FunctionTok{ylab}\NormalTok{(}\StringTok{"Sea Surface Temperature (°C)"}\NormalTok{) }\SpecialCharTok{+}  \CommentTok{\#add labels}
  \FunctionTok{xlab}\NormalTok{(}\StringTok{"Date"}\NormalTok{) }\SpecialCharTok{+}
\FunctionTok{labs}\NormalTok{(}\AttributeTok{title =} \StringTok{"Mean Sea Surface Temperature in Los Angeles, CA"}\NormalTok{, }\AttributeTok{subtitle =} \StringTok{"Maximum, Minimum, \& Mean SST Between 2019 \& 2022 }
\StringTok{(Data: Southern California Coastal Ocean Observation Station, Santa Monica Pier)"}\NormalTok{) }\SpecialCharTok{+}
  \FunctionTok{theme\_bw}\NormalTok{(}\AttributeTok{base\_size =} \DecValTok{9}\NormalTok{) }\CommentTok{\#}

\NormalTok{sst.plot}
\end{Highlighting}
\end{Shaded}

\begin{verbatim}
## `geom_smooth()` using method = 'gam' and formula 'y ~ s(x, bs = "cs")'
\end{verbatim}

\begin{figure}
\includegraphics{../output/unnamed-chunk-4-1} \caption[Sea Surface Temperature Time Series for Los Angeles]{Sea Surface Temperature Time Series for Los Angeles}\label{fig:unnamed-chunk-4}
\end{figure}

\hypertarget{plotting-using-noaa-erdapp-data}{%
\section{Plotting Using NOAA ERDAPP
Data}\label{plotting-using-noaa-erdapp-data}}

For this portion we will use NOAA servers to download monthly mean
temperatures and visualize data using a boxplot. Data can either be
downloaded by using the rerddap package or the NOAA online repository.
We will clean the data by selecting for the year 2021 and then the
longitudes and latitudes of the Southern California Region. Then, we
will average SST per month and create an animated plot to visualize
changes between months.

\href{https://psl.noaa.gov/repository/entry/show?entryid=cac1c2a6-a864-4409-bb77-1fdead8eeb6e}{Data
Linked Here}

Available ERDDAP data can also be pulled into R using the servers()
function

\hypertarget{data-wrangling-1}{%
\subsection{Data Wrangling}\label{data-wrangling-1}}

\begin{Shaded}
\begin{Highlighting}[]
\NormalTok{sst.file }\OtherTok{\textless{}{-}} \FunctionTok{tidync}\NormalTok{(}\StringTok{"/Users/robertdellinger/Documents/Independent\_Project\_Dellinger/Data/sst.mnmean.nc"}\NormalTok{) }
\NormalTok{sst.tibble }\OtherTok{\textless{}{-}} \FunctionTok{hyper\_tibble}\NormalTok{(sst.file,  }\AttributeTok{na.rm =} \ConstantTok{FALSE}\NormalTok{) }\CommentTok{\#turnning nc into tibble}

\NormalTok{sst.data }\OtherTok{\textless{}{-}}\NormalTok{ sst.tibble  }\SpecialCharTok{\%\textgreater{}\%}  
  \FunctionTok{mutate}\NormalTok{(}\AttributeTok{date=}\FunctionTok{as.Date}\NormalTok{(time, }\AttributeTok{origin=}\StringTok{"1800{-}1{-}1 00:00:00"}\NormalTok{),}
         \AttributeTok{day =} \FunctionTok{day}\NormalTok{(date),  }\CommentTok{\# save dates as intergers }
         \AttributeTok{month =} \FunctionTok{month}\NormalTok{(date) }\SpecialCharTok{\%\textgreater{}\%} \FunctionTok{as.integer}\NormalTok{(),}
         \AttributeTok{year =} \FunctionTok{year}\NormalTok{(date) }\SpecialCharTok{\%\textgreater{}\%} \FunctionTok{as.integer}\NormalTok{(),}
         \AttributeTok{sst=}\FunctionTok{as.numeric}\NormalTok{(sst)) }\SpecialCharTok{\%\textgreater{}\%}  \CommentTok{\#converting dates }
  \FunctionTok{filter}\NormalTok{(date }\SpecialCharTok{\textgreater{}=} \FunctionTok{dmy}\NormalTok{(}\DecValTok{010121}\NormalTok{), date }\SpecialCharTok{\textless{}=} \FunctionTok{dmy}\NormalTok{(}\DecValTok{010122}\NormalTok{),}
\NormalTok{         lat }\SpecialCharTok{\textgreater{}} \DecValTok{30}\NormalTok{, lat }\SpecialCharTok{\textless{}} \DecValTok{35}\NormalTok{, }\CommentTok{\#lat limits}
\NormalTok{         lon }\SpecialCharTok{\textgreater{}} \DecValTok{290}\NormalTok{, lon }\SpecialCharTok{\textless{}} \DecValTok{310}\NormalTok{,}
\NormalTok{         year }\SpecialCharTok{==}\DecValTok{2021}\NormalTok{) }\SpecialCharTok{\%\textgreater{}\%} \CommentTok{\#lon limits}
\NormalTok{  dplyr}\SpecialCharTok{::}\FunctionTok{select}\NormalTok{(date, month, lon, lat, sst)}
  
\NormalTok{sst.data }\OtherTok{=} \FunctionTok{as.data.frame}\NormalTok{(sst.data) }\CommentTok{\#convert to data frame}

\DocumentationTok{\#\# group sst based on longitude, latitude and month}
\NormalTok{sst.month }\OtherTok{=}\NormalTok{ sst.data }\SpecialCharTok{\%\textgreater{}\%} 
  \FunctionTok{group\_by}\NormalTok{(lon,lat,month) }\SpecialCharTok{\%\textgreater{}\%} \CommentTok{\# selected for lat, long, and month}
  \FunctionTok{summarise}\NormalTok{(}\AttributeTok{sst =} \FunctionTok{mean}\NormalTok{(sst, }\AttributeTok{na.rm =} \ConstantTok{TRUE}\NormalTok{))}
\end{Highlighting}
\end{Shaded}

\begin{verbatim}
## `summarise()` has grouped output by 'lon', 'lat'. You can override using the
## `.groups` argument.
\end{verbatim}

\begin{Shaded}
\begin{Highlighting}[]
\DocumentationTok{\#\# ungroup grouped data frame}
\NormalTok{sst.month }\OtherTok{=}\NormalTok{ sst.data }\SpecialCharTok{\%\textgreater{}\%} \FunctionTok{ungroup}\NormalTok{()}
\end{Highlighting}
\end{Shaded}

\hypertarget{creating-an-animated-temperature-plot-for-the-southern-california-region}{%
\section{Creating an Animated Temperature Plot for the Southern
California
Region}\label{creating-an-animated-temperature-plot-for-the-southern-california-region}}

Here we use gganimate and ggplot2 to create a animated boxplot using
mean temperature data for the Southern California Region.

\begin{Shaded}
\begin{Highlighting}[]
\NormalTok{animatedplot }\OtherTok{\textless{}{-}} \FunctionTok{ggplot}\NormalTok{(}\AttributeTok{data =}\NormalTok{sst.month, }\CommentTok{\#ggplot sst monthly data}
       \FunctionTok{aes}\NormalTok{(}\AttributeTok{x =}\NormalTok{ month, }\AttributeTok{y =}\NormalTok{ sst, }\AttributeTok{fill =} \FunctionTok{as.factor}\NormalTok{(month)))}\SpecialCharTok{+}
  \FunctionTok{geom\_boxplot}\NormalTok{(}\AttributeTok{show.legend =} \ConstantTok{FALSE}\NormalTok{)}\SpecialCharTok{+} \CommentTok{\#create boxplot}
  \FunctionTok{scale\_fill\_brewer}\NormalTok{(}\AttributeTok{palette =} \StringTok{"Paired"}\NormalTok{)}\SpecialCharTok{+}
\NormalTok{  cowplot}\SpecialCharTok{::}\FunctionTok{theme\_cowplot}\NormalTok{()}\SpecialCharTok{+} 
  \FunctionTok{scale\_x\_continuous}\NormalTok{(}\AttributeTok{breaks =} \DecValTok{1}\SpecialCharTok{:}\DecValTok{12}\NormalTok{)}\SpecialCharTok{+}
  \FunctionTok{ylab}\NormalTok{(}\StringTok{"Sea Surface Temperature (°C)"}\NormalTok{)}\SpecialCharTok{+}  \CommentTok{\#add labels}
  \FunctionTok{xlab}\NormalTok{(}\StringTok{"Month"}\NormalTok{) }\SpecialCharTok{+}
  \FunctionTok{labs}\NormalTok{(}\AttributeTok{title =} \StringTok{"\{closest\_state\}"}\NormalTok{)}\SpecialCharTok{+}
  \FunctionTok{transition\_states}\NormalTok{(month, }\AttributeTok{transition\_length =} \DecValTok{20}\NormalTok{, }\CommentTok{\#transition time and month}
                    \AttributeTok{state\_length =} \DecValTok{2}\NormalTok{, }\AttributeTok{wrap =} \ConstantTok{TRUE}\NormalTok{) }\SpecialCharTok{+} \CommentTok{\#transition wrap}
  \FunctionTok{ease\_aes}\NormalTok{(}\StringTok{\textquotesingle{}linear\textquotesingle{}}\NormalTok{) }\SpecialCharTok{+}
  \FunctionTok{shadow\_mark}\NormalTok{()}
\end{Highlighting}
\end{Shaded}

\begin{Shaded}
\begin{Highlighting}[]
\NormalTok{knitr}\SpecialCharTok{::}\FunctionTok{include\_graphics}\NormalTok{(}\StringTok{"/Users/robertdellinger/Documents/Independent\_Project\_Dellinger/Output/Animated\_Boxplot\_Final.gif"}\NormalTok{)}
\end{Highlighting}
\end{Shaded}

\begin{figure}
\includegraphics{/Users/robertdellinger/Documents/Independent_Project_Dellinger/Output/Animated_Boxplot_Final} \caption[Monthly Variation of Temperature in Southern California Boxplot]{Monthly Variation of Temperature in Southern California Boxplot}\label{fig:unnamed-chunk-5}
\end{figure}

\hypertarget{references}{%
\section{References}\label{references}}

Grolemund, G., \& Wickham, H. (2011). Dates and times made easy with
lubridate. Journal of Statistical Software, 40(3), 1--25. Retrieved from
\url{http://www.jstatsoft.org/v40/i03/}

Kelley, Dan. 2015. Ocedata: Oceanographic Datasets for Oce.
\url{https://CRAN.R-project.org/package=ocedata}.

Kelley, Dan, and Clark Richards. 2018. Oce: Analysis of Oceanographic
Data. \url{https://CRAN.R-project.org/package=oce}.

R Core Team. 2018. R: A Language and Environment for Statistical
Computing. Vienna, Austria: R Foundation for Statistical Computing.
\url{https://www.R-project.org/}.

Richard W. Reynolds, Viva F. Banzon, and NOAA CDR Program (2008): NOAA
Optimum Interpolation 1/4 Degree Daily Sea Surface Temperature (OISST)
Analysis, Version 2. {[}indicate subset used{]}. NOAA National Centers
for Environmental Information. \url{http://doi.org/doi:10.7289/V5SQ8XB5}

Pedersen, T. L., \& Robinson, D. (2017). Gganimate: A grammar of
animated graphics. Retrieved from
\url{http://github.com/thomasp85/gganimate}

Wickham, H. (2016). Ggplot2: Elegant graphics for data analysis.
Springer-Verlag New York. Retrieved from \url{http://ggplot2.org}

Pierce, David. 2017. Ncdf4: Interface to Unidata netCDF (Version 4 or
Earlier) Format Data Files.
\url{https://CRAN.R-project.org/package=ncdf4}.



\end{document}
